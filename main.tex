\documentclass{article}

\usepackage{amsfonts, amsmath,amssymb,amsthm}   % Paquetes de simbología matemática básica
\usepackage{lmodern,microtype,bm}     % Fuente y espaciado entre letras; lo deja bonito
\usepackage{dsfont, graphicx}
\usepackage{mathrsfs, halloweenmath,xcolor}
\usepackage{MnSymbol}
\usepackage{mathtools}
\usepackage{multicol,titlesec}
\usepackage[shortlabels]{enumitem}    % Continuar listas en mini-páginas distintas
\usepackage{physics}
\usepackage[english,spanish]{babel}   % Cambia los comandos de texto predeterminados (capítulos, 			                                        secciones, bibliografía, etc.) a español
\decimalpoint
\usepackage[style=mexican]{csquotes}  % Comillas y otros elementos de citación
\textwidth 16cm                       % Ancho
\oddsidemargin -0.0cm                 % Espacio de margen (como es formato de libro, los margenes se 		                                        declaran para páginas pares e impares
\usepackage[spanish]{todonotes}
\usepackage{hyperref}

\newtheoremstyle{definicion}% name
{3pt}% Space above
{3pt}% Space below
{}% Body font
{}% Indent amount
{\color{blue}\bfseries}% Theorem head font
{.}% Punctuation after theorem head
{.5em}% Space after theorem head
{}%
\theoremstyle{definicion}
\newtheorem{definicion}{Def.}

\theoremstyle{definition}             % Con el paquete amsmath se pueden personalizar los estilos de
\newtheorem*{inst}{Instrucciones}

\theoremstyle{definition}             % Con el paquete amsmath se pueden personalizar los estilos de
\newtheorem{sol}{Solución}

\theoremstyle{definition}
\newtheorem{record}{Recordatorio}

\theoremstyle{definition}
\newtheorem{properties}{Propiedades}

\newtheoremstyle{observacion}% name
{3pt}% Space above
{3pt}% Space below
{}% Body font
{}% Indent amount
{\color{red}\bfseries}% Theorem head font
{.}% Punctuation after theorem head
{.5em}% Space after theorem head
{}%
\theoremstyle{observacion}
\newtheorem{obs}{Obs.}

\theoremstyle{definition}
\newtheorem{prop}{Proposición}

\theoremstyle{plain}
\newtheorem{lemma}{Lema}
\newtheorem{theorem}{Teorema}

\theoremstyle{definition}
\newtheorem{exe}{Ejemplo}

\newtheoremstyle{afirmacion}% name
{3pt}% Space above
{3pt}% Space below
{}% Body font
{}% Indent amount
{\color{green!40!black}\bfseries}% Theorem head font
{.}% Punctuation after theorem head
{.5em}% Space after theorem head
{}%
\theoremstyle{afirmacion}
\newtheorem{corollary}{Corolario}

\newtheoremstyle{notation}% name
{3pt}% Space above
{3pt}% Space below
{}% Body font
{}% Indent amount
{\color{magenta}\bfseries}% Theorem head font
{.}% Punctuation after theorem head
{.5em}% Space after theorem head
{}%
\theoremstyle{notation}
\newtheorem{notation}{Notación}

\theoremstyle{definition}
\newtheorem{eje}{Ejercicio}

\setlength{\parindent}{2em}           % Sangría
\setlength{\parskip}{0.5em}           % Espacio entre párrafos

\title{\Huge{El plano euclidio}}
\author{Geometría Analítica I}
\date{\today}

\begin{document}
    \maketitle

    \section{El espacio vectorial \(\mathbb{R}^{2}\)}

    En esta sección se introduce la herramienta algebraica básica para hacer geometría con parejas, ternas o \(n\)-adas de números.

    \begin{definicion}
        Dados dos vectores \(\vb*{u} = (x, y)\) y \(\vb{v} = (x^{\prime}, y^{\prime})\) en \(\mathbb{R}^{2}\), definimos su \textcolor{blue}{suma vectorial}, o simplemente suma, como el vector \(\vb*{u} + \vb*{v}\) que resulta de sumar componente a componente:

        \begin{equation*}
            \vb*{u} + \vb*{v} \coloneq (x + x^{\prime}, y + y^{\prime}).
        \end{equation*}

        Nótese que en cada coordenada, la suma que se usa es la suma usual de números reales.
    \end{definicion}

    La suma vectorial corresponde geométricamente a la regla del paralelogramo usada para encontrar la resultante de dos vectores. Esto es, se consideran los vectores como segmentos dirigidos que salen del origen y generan entonces un paralelogramo, y el vector que va del origen a la otra esquina es la suma.
    
    \missingfigure[figheight=3cm]{Insertar imagen del método del paralelogramo}

    \begin{definicion}
        Dados un vector \(\vb*{u} = (x, y) \in \mathbb{R}^{2}\) y un número \(t \in \mathbb{R}\) se define la \textcolor{blue}{multiplicación escalar} \(t \vb*{u}\) como el vector que resulta de multiplicar cada componente del vector por el número:

        \begin{equation*}
            t \vb*{u} \coloneq (t x, t y).
        \end{equation*}

        Nótese que en cada coordenada la multiplicación que se usa es de los números reales.
    \end{definicion}

    \begin{obs}
        Notemos que \(t \vb*{u}\) para \(t > 1\) es, estrictamente hablando, una \textcolor{blue}{dilatación} de \(\vb*{u}\), y para \(0 < t < 1\), una \textcolor{blue}{contracción} del mismo. Además, para \(t < 0\), \(t \vb*{u}\) apunta en la dirección contraria, ya que, en particular, \((-1) \vb*{u} \eqqcolon - \vb*{u}\) es el vector que, como segmento dirigido, va del punto \(\vb*{u}\) al \(\vb*{0}\) y el resto se obtiene como dilataciones o contracciones de \(- \vb*{u}\).
    \end{obs}

    Las propiedades básicas de la suma vectorial y la multiplicación escalar se reúnen en el siguiente teorema, donde el vector \(\vb*{0} = (0, \dots, 0)\) es el llamado \textcolor{blue}{vector cero} que corresponde al origen, y, para cada \(x \in \mathbb{R}^{n}\), el vector \(-x \coloneq (-1)x\) se llama \textcolor{blue}{inverso aditivo} de \(x\).

    \begin{theorem}
        Para todos los vectores \(x,\ y,\ z \in \mathbb{R}^{n}\) y para todos los números \(s,\ t \in \mathbb{R}\) se cumple que:

        \begin{enumerate}[label = \textnormal{\Roman*)}]
            \item \((x + y) + z = x + (y + z)\)
            \item \(x + y = y + z\)
            \item \(x + \vb*{0} = x\)
            \item \(x + (-x) = \vb*{0}\)
            \item \(s(tx) = (st)x\)
            \item \(1x = x\)
            \item \(t(x + y) = tx + ty\)
            \item \((s + t)x = sx + tx\)
        \end{enumerate}
    \end{theorem}

    \begin{lemma}
        Si \(x \in \mathbb{R}^{2}\) y \(t \in \mathbb{R}\) son tales que \(tx = \vb*{0}\) entonces \(t = 0\) 0 \(x = \vb*{0}\).
    \end{lemma}
\end{document}